%%%%%%%%%%%%  Generated using docx2latex.com  %%%%%%%%%%%%%%

%%%%%%%%%%%%  v2.0.0-beta  %%%%%%%%%%%%%%

\documentclass[12pt]{article}
\usepackage{amsmath}
\usepackage{latexsym}
\usepackage{amsfonts}
\usepackage[normalem]{ulem}
\usepackage{soul}
\usepackage{array}
\usepackage{amssymb}
\usepackage{extarrows}
\usepackage{graphicx}
\usepackage[backend=biber,
style=numeric,
sorting=none,
isbn=false,
doi=false,
url=false,
]{biblatex}\addbibresource{bibliography.bib}

\usepackage{subfig}
\usepackage{wrapfig}
\usepackage{wasysym}
\usepackage{enumitem}
\usepackage{adjustbox}
\usepackage{ragged2e}
\usepackage[svgnames,table]{xcolor}
\usepackage{tikz}
\usepackage{longtable}
\usepackage{changepage}
\usepackage{setspace}
\usepackage{hhline}
\usepackage{multicol}
\usepackage{tabto}
\usepackage{float}
\usepackage{multirow}
\usepackage{makecell}
\usepackage{fancyhdr}
\usepackage[toc,page]{appendix}
\usepackage[hidelinks]{hyperref}
\usetikzlibrary{shapes.symbols,shapes.geometric,shadows,arrows.meta}
\tikzset{>={Latex[width=1.5mm,length=2mm]}}
\usepackage{flowchart}\usepackage[paperheight=11.0in,paperwidth=8.5in,left=1.0in,right=1.0in,top=1.0in,bottom=1.0in,headheight=1in]{geometry}
\usepackage[utf8]{inputenc}
\usepackage[T1]{fontenc}
\TabPositions{0.5in,1.0in,1.5in,2.0in,2.5in,3.0in,3.5in,4.0in,4.5in,5.0in,5.5in,6.0in,}

\urlstyle{same}

\renewcommand{\_}{\kern-1.5pt\textunderscore\kern-1.5pt}

 %%%%%%%%%%%%  Set Depths for Sections  %%%%%%%%%%%%%%

% 1) Section
% 1.1) SubSection
% 1.1.1) SubSubSection
% 1.1.1.1) Paragraph
% 1.1.1.1.1) Subparagraph


\setcounter{tocdepth}{5}
\setcounter{secnumdepth}{5}


 %%%%%%%%%%%%  Set Depths for Nested Lists created by \begin{enumerate}  %%%%%%%%%%%%%%


\setlistdepth{9}
\renewlist{enumerate}{enumerate}{9}
		\setlist[enumerate,1]{label=\arabic*)}
		\setlist[enumerate,2]{label=\alph*)}
		\setlist[enumerate,3]{label=(\roman*)}
		\setlist[enumerate,4]{label=(\arabic*)}
		\setlist[enumerate,5]{label=(\Alph*)}
		\setlist[enumerate,6]{label=(\Roman*)}
		\setlist[enumerate,7]{label=\arabic*}
		\setlist[enumerate,8]{label=\alph*}
		\setlist[enumerate,9]{label=\roman*}

\renewlist{itemize}{itemize}{9}
		\setlist[itemize]{label=$\cdot$}
		\setlist[itemize,1]{label=\textbullet}
		\setlist[itemize,2]{label=$\circ$}
		\setlist[itemize,3]{label=$\ast$}
		\setlist[itemize,4]{label=$\dagger$}
		\setlist[itemize,5]{label=$\triangleright$}
		\setlist[itemize,6]{label=$\bigstar$}
		\setlist[itemize,7]{label=$\blacklozenge$}
		\setlist[itemize,8]{label=$\prime$}

\setlength{\topsep}{0pt}\setlength{\parskip}{8.04pt}
\setlength{\parindent}{0pt}

 %%%%%%%%%%%%  This sets linespacing (verticle gap between Lines) Default=1 %%%%%%%%%%%%%%


\renewcommand{\arraystretch}{1.3}


%%%%%%%%%%%%%%%%%%%% Document code starts here %%%%%%%%%%%%%%%%%%%%



\begin{document}
Mark Demore, 2d Lt\par

CSCE686 – Dr. Lamont\par

Project Proposal\par


\vspace{\baselineskip}
\tab A real world problem that incorporates two different NP-Complete problem models is a airlift scheduling program for Air Mobility Command. While there are certainly many other considerations in the actual application of such a program, it can be simplified to a combination of the Knapsack problem and the Vehicle Routing problem. For example, the knapsack problem can be used to determine how to load the planes, maximizing the value of cargo that can fit in each aircraft, and the vehicle routing problem can be used to determine the order in which to drop off the cargo.\par

The Knapsack Problem can be defined as:\par

Let \textit{S} be a set of \textit{n} items, each with a weight \textit{w\textsubscript{i}} and a value \textit{v\textsubscript{i}}, and paired with a maximum cargo weight \textit{W}. Constraints:  \(  \sum _{i=1}^{n}w_{i}  \leq W \) . Objective: maximize value.\par

The Vehicle Routing Problem can be defined as:\par

Let \textit{G} = (\textit{V}, \textit{A}) be a graph where \textit{V} = $ \{ $ 1, $ \ldots $ , \textit{n}$ \} $  is a set of vertices representing air drop locations with the airfield located at vertex 1, and \textit{A} is the set of arcs. With every arc (\textit{i, j}) \textit{i} =/=\textit{j} is associated a non-negative cost matrix \textit{C} = (\textit{c\textsubscript{ij}}). Constraints: (i) each air drop in \textit{V}$\textbackslash$ $ \{ $ 1$ \} $  is visited exactly once by exactly one plane; (ii) all flight plans start and end at the airfield. Objective: minimize cost.\par

These problems can be combined in this context as:\par

Let \textit{G} = (\textit{V}, \textit{A}) be a graph where \textit{V} = $ \{ $ 1, $ \ldots $ , \textit{n}$ \} $  is a set of vertices representing air drop locations with the airfield located at vertex 1, and \textit{A} is the set of arcs between them. Every arc (\textit{i, j}) \textit{i} =/=\textit{j} is associated a non-negative cost matrix \textit{C} = (\textit{c\textsubscript{ij}}). Let each vertex be assigned a set \textit{S},\textit{ }of \textit{n} items, each with a weight \textit{w\textsubscript{i}} and a value \textit{v\textsubscript{i}}. Let P be a set of planes, each with a maximum cargo weight \textit{W\textsubscript{i}}. Constraints:  (i) each air drop in \textit{V}$\textbackslash$ $ \{ $ 1$ \} $  is visited exactly once by exactly one plane; (ii) all flight plans start and end at the airfield; (iii)  \(  \sum _{i=1}^{n}w_{i}  \leq W_{P} \) . Objective: maximize total value and minimize total cost across all flight plans.\par

\tab The search landscape for these problems entail computing the knapsack problem and the vehicle routing problem for each aircraft, for each combination of airdrop locations and of items at each location. A heuristic will need to be implemented to determine the tradeoff balance between the two objectives of minimizing cost and maximizing value and also for prioritizing which planes (if they are of varying cargo capacity) should be planned first.\par

\textit{D\textsubscript{i}}: G = graph of vertices of air drop locations and arcs, C = cost matrix for each arc, P = set of available aircraft, S = set of items needed at each location\par

\textit{D\textsubscript{o}}: R = set of sets of routes for each plane, L = set of sets of loads for each plane\par


\printbibliography
\end{document}